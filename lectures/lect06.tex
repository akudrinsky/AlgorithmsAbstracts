\section{Продолжение строк.}
\subsection{Алгоритм Ахо -- Корасик для нахождения всех вхождений набора подстрок.}%

\begin{Def}
	\textbf{Бор} --- дерево, в котором каждая вершина обозначает строку, а каждое ребро обозначает букву. Строка, соответсвующая вершине (то есть в ней заканчивающаяся), получается конкатенацией всех букв, соответсвующих ребрам пути из корня в эту самую вершину. По определению, корню бора соответствует пустая строка.
\end{Def}

\begin{Def}
	\textbf{Терминальная вершина} бора для набора слов --- вершина, которой соответствует слово из этого набора.
\end{Def}

\subsubsection{Построение бора.}%
\label{ssub:Построение бора.}

Для добавления строки в бор мы: \\
Прочитав очередной символ в цикле по строке, переходим по соответсвующему ему ребру или создаем такое ребро если потребуется. \\
После завершения строки помечаем последнюю вершину как терминальную.

Таким образом, бор строится за линейное по сумме всех строк в наборе время.

\subsubsection{Преобразование бора в автомат.}%
Будем понимать вершины бора и соответсвующие им строки как состояния конечного детерменированного автомата. Однако мы сталкиваемся с проблемой, ребер бора не достаточно для отражения всех возможных переходов между состояниями автомата. 
\begin{example}
	Даю установку: Вы видите картинку бора для строк adam и dprk.
	Ребра бора не отражают тот факт, что перейдя под воздействием символов $"ad"$ в некоторое состояние  $S_1$ мы все еще можем перейти в состояние  $S_2$.
\end{example}

\begin{Def}
	\textbf{Суффиксная ссылка} вершины $v$ --- ссылка (мнимая стрелка в боре) на вершину $u$, такую, что состояние $u$ --- наибольший собственный суффикс состояния $v$, а если такой вершины  $u$ нет, то ссылка на корень. По определению, ссылка из корня ведет в корень.
\end{Def}

\begin{example}
	Для бора строк $"adam"$ и $"dprk"$ единственной суффиксной ссылкой, не ведущей в корень, будет ссылка из состояния $"ad"$, в состояние $"d"$, зачеркнутая на рисунке выше.
\end{example}

\textbf{Чтобы найти суффиксную ссылку} для вершины $v$: 
\begin{enumerate}
	\item Если $v$ -- корень, то его ссылка указывает на корень.
	\item Иначе: \\  
		  Пусть $p$ --- родитель  $v$, а  $c$ --- буква, переход по которой привел из  $p$ в  $v$. \\
		\begin{enumerate}
			\item Если из суффиксной ссылки вершины $p$ есть ребро $c$: \\
		  		  Cуффиксная ссылка вершины $v$ --- вершина, в которую автомат придет из суффиксной ссылки вершины $p$, пройдя по ребру $c$. \\
		  		  Ведь суффиксная ссылка $p$ указывает на наибольший собственный суффикс родителя, к которому мы добавляем символ $c$, дополняющий суффикс родителя до суффикса  $v$. \\
			%\item Если $p$ --- корень.
			\item Иначe: \\
				  
		\end{enumerate}
\end{enumerate}

