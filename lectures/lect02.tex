\section{Потоки}%
\label{sec:Потоки}

\begin{Def}
	\textbf{Сеть} -- ориентированный граф $G = (V, E)$, с двумя выделенными различными вершинами: $s$ -- исток и  $t$ -- сток, а также с функцией $cap: E \rightarrow \Z_{>0}$.
\end{Def}

\begin{Def}
	\textbf{Поток} в сети -- функция $f: V \times V \rightarrow \Z$:
	
	\begin{enumerate}
		\item $\forall u, v \in V \hookrightarrow f(u, v) \leq cap(u, v)$
		\item $\forall v \in V \setminus \{s, t\} \hookrightarrow \sum\limits_{\substack{u \\ (u, v) \in E}} f(u, v) = \sum\limits_{\substack{w \\ (v, w) \in E}} f(v, w)$
		\item $\forall u, v \in V \hookrightarrow f(u, v) = -f(v, u)$
	\end{enumerate}

	\begin{note}
		Свойства 2 и 3 можно объединить в следующем виде $\forall v \in V \hookrightarrow \sum\limits_{u \in V} f(v, u) = 0$
	\end{note}
\end{Def}


\begin{Def}
	\textbf{Остаточная сеть} $G_f$ сети  $G$ с потоком $f$ -- сеть, у которой модифицируются пропускные способности:  $cap_f(u, v) = cap(u, v) - f(u, v)$, и удаляются ребра с нулевой  $cap_f$.
\end{Def}

\begin{Def}
	\textbf{Величина потока} $\lvert f \rvert$-- то сколько "вытекает"\ из $s$ -- сумма исходящих из  $s$ потоков.
\end{Def}

\begin{Def}
	\textbf{Разрез} сети $G$ -- пара подсетей $(S, T)$:  $s \in S, t \in T, S \cap T = \emptyset, S \cup T = G$ \\
	$cap(S, T) = \sum\limits_{\substack{u \in S \\ v \in T}} cap(u, v)$  \\
	$f(S, T) = \sum\limits_{\substack{u \in S \\ v \in T}} f(u, v)$ 
\end{Def}

\begin{lemma}
	$(S, T)$ -- разрез $\Rightarrow$ $\lvert f \rvert = f(S, T)$
\end{lemma}

\begin{proof}
	На семинаре.	
\end{proof}

\begin{lemma}
	$(S, T)$ -- разрез $\Rightarrow$ $f(S, T) \leq cap(S, T)$
\end{lemma}

\begin{proof}
	$\forall u, v \in V \hookrightarrow f(u, v) \leq cap(u, v) \Rightarrow$ чтд.
\end{proof}

\begin{theorem} (Форда -- Фалкерсона)
	Следующие утверждения эквивалентны:
	\begin{enumerate}
		\item $f$ -- max поток в $G$  
		\item в $G_f$ нет пути из  $s$ в  $t$
		\item $\lvert f \rvert$  = $cap$ некоторого разреза.
	\end{enumerate}
\end{theorem}

\begin{proof} \ \\
	\underline{$1 \Rightarrow 2$}: От противного: \\
		Пусть  $f$ -- max, но в  $G_f$ есть путь  $s \rightarrow t$. \\
		Рассмотрим  $x = min(cap_f) > 0$ на этом пути. \\
		Значит вдоль пути можно пустить  $x$ единиц потока  $\Rightarrow$ в исходной сети можно было пустить на x единиц больше $\Rightarrow$ f -- не  max. 

		\underline{$2 \Rightarrow 3$}: \\
		Пусть $S$ = множество вершин, доступных из  $s$ в  $G_f$,  $t \notin S$ \\
		$T = V \setminus S$. \\
		По определению $(S, T)$ -- разрез. \\
		Покажем что $\lvert f \rvert = cap(S, T) = \sum\limits_{\substack{u \in S \\ v \in T}} cap(u, v)$. \\
		Рассмотрим ребро $(u, v)$: $u \in S, v \in T$ \\
		$u$ -- доступно в  $G_f$,  $v$ -- нет  $\Rightarrow$ оно удаляется в остаточной сети $\Rightarrow \\
		f(u, v) = cap(u, v) \Rightarrow \sum\limits_{\substack{u \in S \\ v \in V}} cap(u, v) = \sum\limits_{\substack{u \in S \\ v \in V}} f(u, v) \Rightarrow$ \\
		$f(S, T) = \lvert f \rvert = cap(S, T)$. 

		\underline{$3 \Rightarrow 1$}: \\
	$\exists (S, T):\ \lvert f \rvert = cap(S, T)$. \\
	При этом $\forall (S, T)$ -- разрез $\hookrightarrow f(S, T) \leq cap(S, T) \Rightarrow f$  -- max. 
\end{proof}

\subsection{Алгоритм Форда -- Фалкерсона для поиска максимального потока.}
\begin{enumerate}
	\item Изначально поток равен 0.
	\item Пока в $G_f$ есть путь из  $s$ в  $t$:
		\begin{enumerate}
			\item $x = min(cap_f$ на этом пути  $) > 0$ 
			\item Увеличим поток $f$ на найденом пути на $x$ и изменим остаточную сеть соответствующим образом.
		\end{enumerate}
\end{enumerate}
$\lvert f \rvert$ увеливается на целое положительное число и ограничен сверху, значит алгоритм закончится.
Асимтотика -- $O(ans \times \lvert E  \rvert)$  \\

\subsection{Алгоритм Эдмондса -- Карпа для поиска максимального потока.}
Та же самая идея, только теперь вместо произвольного пути будем выбирать кратчайший путь по ребрам. По сути, DFS меняется на BFS.
\begin{enumerate}
	\item Изначально поток равен 0.
	\item Пока в $G_f$ есть \textit{минимальный} путь из $s $ в $t$:
		\begin{enumerate}
			\item $x = min(cap_f$ на этом пути $) > 0$.
			\item  увеличим поток $f$ на  $x$  и изменим остаточную сеть соответствующим образом.
		\end{enumerate}
\end{enumerate}

Алгоритм будет иметь асимптотику $O(\lvert V \rvert \times \lvert E \rvert ^ 2)$ \\

Докажем это.
\begin{Def}
	$dist(u, v)$ -- минимальное число ребер между вершинами.
\end{Def}

\begin{lemma}
	Пусть поток $f'$ получается из потока $f$ после одной итерации алгоритма Эдмондса -- Карпа. \\
	Пусть $dist'(u, v)$ -- кратчайшее расстояние между ребрами в $G_{f'}$, тогда
	$\forall v \in V \{s, t\} \hookrightarrow dist'(s, v) \geq dist(s, v)$.  \\
	То есть с каждой итерацией расстояние не убывает.
\end{lemma}

\begin{proof} От противного: \\
	Пусть $\exists v \in V \setminus \{s, t\}:\ dist'(s, v) < dist(s, v)$ и при этом $dist(s, v)$ -- минимальное возможное \\
	Пусть $u$ -- предыдущая вершина перед  $v$	на кратчайшем пути из $s$ в  $v$ в $G_{f'}$ \\
	Тогда $dist'(s, u) = dist'(s, v) - 1$ (по определению dist). \\
	Также $dist'(s,u) \geq dist(s, u)$ ведь мы выбрали минимально удаленное $v$.\\
	Рассмотрим 2 случая: \\
		$(u, v) \in E_f$, тогда: \\
			$dist(s, v) \leq dist(s, u) + 1 \leq dist'(s, u) + 1 \leq dist'(s, v)$ \\
			Но при этом $dist'(s, v) < dist(s, v)$ \\
			Противоречие.
		
			$(u, v) \notin E_f$, но известно что $(u, v) \in E_{f'}$.\\
			Появление ребра $(u,v)$ после увеличения потока означает увеличение потока по обратному ребру $(v,u)$. Увеличение потока производится вдоль кратчайшего пути, а значит вершина  $v$ -- предыдущая перед  $u$ на пути из  $s$ в  $u$\\
			Это значит что  $dist(s, v) = dist(s, u) - 1 \leq dist'(s, u) - 1 = dist'(s, v) - 2$, но ведь $dist'(s, v) < dist(s, v)$. Противоречие.

\end{proof}

\begin{Def}
	\textbf{Насыщенное} -- ребро, вдоль которого идет поток, равный его capacity.
\end{Def}

\begin{lemma}
	Любое ребро сети насыщается не более $O(\lvert V \rvert)$ раз, то есть в алгоритме $O(\lvert V \rvert \lvert E \rvert)$ итераций, то есть каждая итерация насыщает хоть 1 ребро.
\end{lemma}

\begin{proof}\ \\
	Рассмотрим ребро $(u, v)$ в момент его насыщения. \\
	Если оно насыщается, то оно лежит на кратчайшем пути от $s$ до  $v$ (так работает наш алгоритм).\\
	Значит  $dist(s, u) + 1 = dist(s, v).$ \\

	Теперь посмотрим, когда оно могло насытиться еще один раз: \\
	Оно сначала должно перестать быть насыщенным, для этого нужно отменить поток вдоль $(u, v)$, то есть пустить поток вдоль  $(v, u)$ \\
	Пусть  $dist'$ -- расстояние в момент, когда пускается поток вдоль  $(v, u)$. \\
	$dist'(s, v) + 1 = dist'(s, u)$ (Так как проталкивание происходит вдоль кратчайшего пути). \\
	Со временем расстояния только увеличиваются, значит $dist'(s, v) \geq dist(s, v) + 1 = dist(s, u) + 2$ \\
	Таким образом от момента насыщения, до момента повторного насыщения расстояние  $dist(s, u)$ должно вырасти по меньшей мере на 2. \\
	Однако все расстояния ограничены $O(\lvert V \rvert)$, значит и насыщений может быть только $O(\lvert V \rvert)$.
\end{proof}
