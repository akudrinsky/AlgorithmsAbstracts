\section{Суффиксное дерево. Алгоритм Укконена.}%
\label{sec:Суффиксное дерево. Алгоритм Укконена.}
$S$ --- строка \\
$S_i$ --- символ на  $i$-той позиции строки. Индексы будем считать с единицы. \\
$S^i$ --- суффикс строки, начинающийся с $i$-той позиции и включающий ее.

\subsection{Суффиксное дерево.}%
\label{sub:Суффиксное дерево.}

\begin{Def}
   \textbf{Проходная} --- вершина с исходящей степенью равной 1.
\end{Def}

Чтобы построить суффиксное дерево, припишем к исходной строке $S$ символ \$, и добавим все суффиксы получившейся строчки в бор, а за тем \textit{сожмем проходные вершины}, устранив вершину и склеив ее два ребра в одно. Заметим, что на каждом шаге добавлялась не более чем одна терминальная и не более чем одна обычная вершина, значит всего вершин $O(n)$.
Для оптимизации памяти, вместо строк будем хранить индекс начала и индекс конца подстроки в $S$.

Бор хранит в себе все префиксы добавленных строк, значит получившееся дерево содержит все префиксы всех суффиксов, то есть \textit{содержит все возможные подстроки $S$}.

\begin{example}
   Даю установку вы видите рисунок. 
\end{example}

\subsubsection{Поиск суффиксных ссылок.}
%Тут надо бы сделать ссылку на определение.
Добавим в получившийся бор суффиксные ссылки. Так как бор содержит все подстроки, то суффиксная ссылка строки отличается от самой строки только отсутствием первого символа. Значит мы можем легко ее посчитать, перейдя из корня по символам кроме первого. \\
Тогда можем заметить, что суффиксная ссылка листа --- лист, ведь каждый листо соответствует суффиксу исходной строки. Из аналогичных рассуждений, ссылка из нетерминальной вершины ведет в нетерминальную вершину. 

\begin{prop}
    Суффиксная ссылка из вершины не может вести в середину сжатого ребра. 
\end{prop}

\begin{proof} \ \\
    \begin{enumerate}
        \item \underline{Если вершина терминальная:} 
            то ее суффиксная ссылка --- терминальная вершина $\Rightarrow$ не в середине ребра.

        \item \underline{Если вершина нетерминальная и не корень:}
            то она имеет хотя бы 2 ребенка. Значит строку, соответствующую данной вершине, назовем ее $\alpha$, можно в исходном тексте продлить двумя символами, $c_1$ и $c_2$. 
            То есть в тексте есть два вхождения $\alpha c_1$ и $\alpha c_2$.
            У каждого из вхождений можем отбросить первый символ и продлить символами $c_1$ $c_2$. Значит у суффиксной ссылки данной вершины должны быть переходы по $c_1$ и $c_2$. Значит ссылка не может быть серединой ребра. 

        \item \underline{Если вершина --- корень:} то ее суффиксной ссылкой по определению является корень.
    \end{enumerate}
\end{proof}

Однако строка может заканчиваться не в вершине, а в середине ребра. Тогда чтобы найти суффиксную ссылку строки, найдем ссылку вершины, соответствующей наибольшему префиксу данной строки и пройдем из нее по оставшимся символам строки. Это возможно, так как в силу структуры дерева, ссылку любой вершины можно продлить по тем же символам что и исходную вершину.

\subsection{Алгоритм Укконена для построения суффиксного дерева.}
Пусть $S = "abaabab\$"$\\
Построим дерево вместе с суффиксными ссылками следующим образом: \\
Будем проходить по $S$ слева направо и расширять строку текущим символом, достраивая дерево. 
В каждый момент времени мы храним дерево для уже обработанного префикса строки. 

Когда мы получаем новый символ $c$, каждый суффикс строки увеличивается на этот самый символ. 
%Если бы у нас был обычный, а не сжатый бор, мы бы просто добавили к суффиксу переход по $c$ и перенесли терминальность на новую вершину, а затем перешли бы по суффиксной ссылке и проделали то же самое. В определенный момент мы пришли бы к вершине, из которой уже имелся бы переход по $c$, значит у нее уже нет необходимости добавлять переход, достаточно только переставить терминальность. Однако наш бор сжатый, значит нам придется переходить по суффиксной ссылке из середины ребра, а это мы делать уже предусмотрительно умеем. 

При добавление символа $c$, все суффиксы в порядке убывания длины разбиваются на три категории, внутри каждой строки идут подряд. 
\begin{enumerate}
    \item Уже являющиеся листьями.
    \item Нелисться, из которых еще не было перехода по $c$.
    \item Нелисться, из которых уже был переход по $c$.
\end{enumerate}

Тогда 
\begin{itemize}
    \item Если вершина уже лист, то при продлении к нeму будет дописываться один символ, это процесс можно ускорить, дописав в строку, соответствующую этому листу исходную строку $S$ до самого конца. 
    \item Если перехода по $c$ еще не было, его нужно создать, а новосозданная вершины станет листом, значит в ребре, ведущем к ней, сразу допишем всю оставшуюся часть  $S$, а не только  $c$.
    \item Если у вершины уже есть переход по $c$, то мы просто перейдем по  $c$ и перенесем терминальность.
\end{itemize}

Таким образом с листьями мы быстро разбираемся, дополняя их до конца. \\
Будем хранить указатель, на самый длинный суффикс, не являющийся листом. Создаем в нем переход по $c$, который станет листом. Вполне возможно что этот указатель ведет в середину ребра, и чтобы добавить вершину, нам придется это ребро расщeпить. Затем перейдем по суффиксной ссылке и проделаем то же самое с ней. Продолжим так пока не окажемся в вершине, в которой уже был переход по $c$, перейдя из нее по $c$ мы и окажемся в вершине, которая теперь является самым длинным суффиксом и нелистом.

\begin{lstlisting}
cur = pointer to the longest non-leaf suffix.
for c in s:
    while (cur->son[c] == nullptr):
        If necessery, split current edge.
        cur->son[c] = new Node(all suffix til string end);
        cur = suflink(cur)

    cur = go(cur, b)
\end{lstlisting}

У всех создаваемых вершин, кроме терминальных будем рассчитывать суффиксные ссылки.
\subsubsection{Асимптотика.}
Переход по суффиксной ссылке делается $O(n)$ раз, ведь после перехода по суффиксной ссылке мы прыгаем на другую ветку, забывая о старой. Значит спускаемся вниз по символам тоже не более чем $O(n)$ раз.
Как мы уже знаем, в процессе работы алгоритма создается $O(n)$ новых вершин. 
Таким образом алгоритм работает за $O(n)$.

